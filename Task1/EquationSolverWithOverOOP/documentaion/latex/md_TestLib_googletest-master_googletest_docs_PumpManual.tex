{\bfseries P}ump is {\bfseries U}seful for {\bfseries M}eta {\bfseries P}rogramming.

\section*{The Problem}

Template and macro libraries often need to define many classes, functions, or macros that vary only (or almost only) in the number of arguments they take. It\textquotesingle{}s a lot of repetitive, mechanical, and error-\/prone work.

Variadic templates and variadic macros can alleviate the problem. However, while both are being considered by the C++ committee, neither is in the standard yet or widely supported by compilers. Thus they are often not a good choice, especially when your code needs to be portable. And their capabilities are still limited.

As a result, authors of such libraries often have to write scripts to generate their implementation. However, our experience is that it\textquotesingle{}s tedious to write such scripts, which tend to reflect the structure of the generated code poorly and are often hard to read and edit. For example, a small change needed in the generated code may require some non-\/intuitive, non-\/trivial changes in the script. This is especially painful when experimenting with the code.

\section*{Our Solution}

Pump (for Pump is Useful for Meta Programming, Pretty Useful for Meta Programming, or Practical Utility for Meta Programming, whichever you prefer) is a simple meta-\/programming tool for C++. The idea is that a programmer writes a {\ttfamily foo.\+pump} file which contains C++ code plus meta code that manipulates the C++ code. The meta code can handle iterations over a range, nested iterations, local meta variable definitions, simple arithmetic, and conditional expressions. You can view it as a small Domain-\/\+Specific Language. The meta language is designed to be non-\/intrusive (s.\+t. it won\textquotesingle{}t confuse Emacs\textquotesingle{} C++ mode, for example) and concise, making Pump code intuitive and easy to maintain.

\subsection*{Highlights}


\begin{DoxyItemize}
\item The implementation is in a single Python script and thus ultra portable\+: no build or installation is needed and it works cross platforms.
\item Pump tries to be smart with respect to \href{https://github.com/google/styleguide}{\tt Google\textquotesingle{}s style guide}\+: it breaks long lines (easy to have when they are generated) at acceptable places to fit within 80 columns and indent the continuation lines correctly.
\item The format is human-\/readable and more concise than X\+ML.
\item The format works relatively well with Emacs\textquotesingle{} C++ mode.
\end{DoxyItemize}

\subsection*{Examples}

The following Pump code (where meta keywords start with {\ttfamily \$}, {\ttfamily \mbox{[}\mbox{[}} and {\ttfamily \mbox{]}\mbox{]}} are meta brackets, and {\ttfamily \$\$} starts a meta comment that ends with the line)\+:


\begin{DoxyCode}
$var n = 3     $$ Defines a meta variable n.
$range i 0..n  $$ Declares the range of meta iterator i (inclusive).
$for i [[
               $$ Meta loop.
// Foo$i does blah for $i-ary predicates.
$range j 1..i
template <size\_t N $for j [[, typename A$j]]>
class Foo$i \{
$if i == 0 [[
  blah a;
]] $elif i <= 2 [[
  blah b;
]] $else [[
  blah c;
]]
\};

]]
\end{DoxyCode}


will be translated by the Pump compiler to\+:


\begin{DoxyCode}
// Foo0 does blah for 0-ary predicates.
template <size\_t N>
class Foo0 \{
  blah a;
\};

// Foo1 does blah for 1-ary predicates.
template <size\_t N, typename A1>
class Foo1 \{
  blah b;
\};

// Foo2 does blah for 2-ary predicates.
template <size\_t N, typename A1, typename A2>
class Foo2 \{
  blah b;
\};

// Foo3 does blah for 3-ary predicates.
template <size\_t N, typename A1, typename A2, typename A3>
class Foo3 \{
  blah c;
\};
\end{DoxyCode}


In another example,


\begin{DoxyCode}
$range i 1..n
Func($for i + [[a$i]]);
$$ The text between i and [[ is the separator between iterations.
\end{DoxyCode}


will generate one of the following lines (without the comments), depending on the value of {\ttfamily n}\+:


\begin{DoxyCode}
Func();              // If n is 0.
Func(a1);            // If n is 1.
Func(a1 + a2);       // If n is 2.
Func(a1 + a2 + a3);  // If n is 3.
// And so on...
\end{DoxyCode}


\subsection*{Constructs}

We support the following meta programming constructs\+:

\tabulinesep=1mm
\begin{longtabu} spread 0pt [c]{*{2}{|X[-1]}|}
\hline
\rowcolor{\tableheadbgcolor}\textbf{ {\ttfamily \$var id = exp} }&\textbf{ Defines a named constant value. {\ttfamily \$id} is valid util the end of the current meta lexical block.  }\\\cline{1-2}
\endfirsthead
\hline
\endfoot
\hline
\rowcolor{\tableheadbgcolor}\textbf{ {\ttfamily \$var id = exp} }&\textbf{ Defines a named constant value. {\ttfamily \$id} is valid util the end of the current meta lexical block.  }\\\cline{1-2}
\endhead
{\ttfamily \$range id exp..exp} &Sets the range of an iteration variable, which can be reused in multiple loops later. \\\cline{1-2}
{\ttfamily \$for id sep \mbox{[}\mbox{[} code \mbox{]}\mbox{]}} &Iteration. The range of {\ttfamily id} must have been defined earlier. {\ttfamily \$id} is valid in {\ttfamily code}. \\\cline{1-2}
{\ttfamily \$(\$)} &Generates a single {\ttfamily \$} character. \\\cline{1-2}
{\ttfamily \$id} &Value of the named constant or iteration variable. \\\cline{1-2}
{\ttfamily } &Value of the expression. \\\cline{1-2}
{\ttfamily \$if exp \mbox{[}\mbox{[} code \mbox{]}\mbox{]} else\+\_\+branch} &Conditional. \\\cline{1-2}
{\ttfamily \mbox{[}\mbox{[} code \mbox{]}\mbox{]}} &Meta lexical block. \\\cline{1-2}
{\ttfamily cpp\+\_\+code} &Raw C++ code. \\\cline{1-2}
{\ttfamily \$\$ comment} &Meta comment. \\\cline{1-2}
\end{longtabu}
{\bfseries Note\+:} To give the user some freedom in formatting the Pump source code, Pump ignores a new-\/line character if it\textquotesingle{}s right after {\ttfamily \$for foo} or next to {\ttfamily \mbox{[}\mbox{[}} or {\ttfamily \mbox{]}\mbox{]}}. Without this rule you\textquotesingle{}ll often be forced to write very long lines to get the desired output. Therefore sometimes you may need to insert an extra new-\/line in such places for a new-\/line to show up in your output.

\subsection*{Grammar}


\begin{DoxyCode}
code ::= atomic\_code*
atomic\_code ::= $var id = exp
    | $var id = [[ code ]]
    | $range id exp..exp
    | $for id sep [[ code ]]
    | $($)
    | $id
    | $(exp)
    | $if exp [[ code ]] else\_branch
    | [[ code ]]
    | cpp\_code
sep ::= cpp\_code | empty\_string
else\_branch ::= $else [[ code ]]
    | $elif exp [[ code ]] else\_branch
    | empty\_string
exp ::= simple\_expression\_in\_Python\_syntax
\end{DoxyCode}


\subsection*{Code}

You can find the source code of Pump in \href{../scripts/pump.py}{\tt scripts/pump.\+py}. It is still very unpolished and lacks automated tests, although it has been successfully used many times. If you find a chance to use it in your project, please let us know what you think! We also welcome help on improving Pump.

\subsection*{Real Examples}

You can find real-\/world applications of Pump in \href{https://github.com/google/googletest/tree/master/googletest}{\tt Google Test} and \href{https://github.com/google/googletest/tree/master/googlemock}{\tt Google Mock}. The source file {\ttfamily foo.\+h.\+pump} generates {\ttfamily foo.\+h}.

\subsection*{Tips}


\begin{DoxyItemize}
\item If a meta variable is followed by a letter or digit, you can separate them using {\ttfamily \mbox{[}\mbox{[}\mbox{]}\mbox{]}}, which inserts an empty string. For example {\ttfamily Foo\$j\mbox{[}\mbox{[}\mbox{]}\mbox{]}Helper} generate {\ttfamily Foo1\+Helper} when {\ttfamily j} is 1.
\item To avoid extra-\/long Pump source lines, you can break a line anywhere you want by inserting {\ttfamily \mbox{[}\mbox{[}\mbox{]}\mbox{]}} followed by a new line. Since any new-\/line character next to {\ttfamily \mbox{[}\mbox{[}} or {\ttfamily \mbox{]}\mbox{]}} is ignored, the generated code won\textquotesingle{}t contain this new line. 
\end{DoxyItemize}